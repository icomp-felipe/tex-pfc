% -----------------------------------------------------------
% Universidade Federal do Amazonas - UFAM
% Faculdade de Tecnologia - FT
% Engenharia da Computação - FT05
% Período Letivo 2023/01
% -----------------------------------------------------------
% Monografia de Conclusão de Curso entitulada:
% IsenSys - Processador de Solicitações de Isenção de
%           Taxa de Inscrição em Concursos Públicos
% por Felipe André <felipeandre@ufam.edu.br>
% -----------------------------------------------------------
% Utilizando o modelo de pacote 'abntex2', adaptado por
% Roberto Cidade Fonseca, do Instituto de Computação da UFAM,
% disponível em: https://icomp.ufam.edu.br/normas-para-tcc/modelos-de-monografia.html
% -----------------------------------------------------------

\documentclass[
	12pt,			% tamanho da fonte
	openright,		% capítulos começam em página ímpar (insere página vazia caso preciso)
	oneside,	
	a4paper,		% tamanho do papel
	english,		% idioma adicional para hifenização
	brazil			% o último idioma é o principal do documento
]{abntex2/abntex2}  % Entre chaves vai o caminho para o arquivo .cls

% ----------| Pacotes básicos |----------
\usepackage{bookman}			% Usa a fonte 'Bookman Old Style'
\usepackage[T1]{fontenc}		% Seleção de códigos de fonte
\usepackage[utf8]{inputenc}		% Codificação do documento (conversão automática dos acentos)
\usepackage{color}				% Controle das cores
\usepackage{graphicx}			% Inclusão de gráficos
\usepackage{microtype} 			% para melhorias de justificação

\usepackage[brazilian,hyperpageref]{backref}  % Páginas com as citações na bibl
\usepackage[alf]{abntex2cite}                 % Citações no padrão ABNT

% ----------| Preâmbulo |----------
\titulo{IsenSys - Processador de Solicitações de Isenção de Taxa de Inscrição em Concursos Públicos}
\autor{Felipe André Souza da Silva}
\local{Manaus - AM}
\data{Outubro de 2023}
\orientador[Orientador:]{Dr. Edson Nascimento Silva Júnior}
\instituicao{Universidade Federal do Amazonas}
\curso{Bacharelado em Engenharia da Computação}
\tipotrabalho{Monografia}
\preambulo{Monografia de Graduação apresentada à Faculdade de Tecnologia da UFAM como requisito parcial para a obtenção do grau de bacharel em Engenharia da Computação.}

% ----------| Informações do PDF |----------
\makeatletter
\hypersetup{
	pdftitle={\@title}, 
	pdfauthor={\@author},
    pdfsubject={\imprimirpreambulo},
    pdfcreator={LaTeX with abnTeX2},
	pdfkeywords={abnt}{latex}{abntex}{abntex2}{trabalho acadêmico},
	bookmarksdepth=4
}
\makeatother

% ----------| Espaçamentos entre linhas e parágrafos |----------
\setlength{\parindent}{1.3cm}

% ----------| Compila o Índice |----------
\makeindex

% ----------| Início do Documento |----------
\begin{document}

\noindent

% Seleciona o idioma do documento (conforme pacotes do babel)
\selectlanguage{brazil}

% Retira espaço extra obsoleto entre as frases
\frenchspacing

% Compila a Capa
\imprimircapa

% Compila a Folha de rosto (o * indica que haverá a ficha bibliográfica)
\imprimirfolhaderosto*

% Folha de aprovação
%
% Isto é um exemplo de Folha de aprovação, elemento obrigatório da NBR
% 14724/2011 (seção 4.2.1.3). Você pode utilizar este modelo até a aprovação
% do trabalho. Após isso, substitua todo o conteúdo deste arquivo por uma
% imagem da página assinada pela banca com o comando abaixo:
%
% \includepdf{folhadeaprovacao_final.pdf}
%
\begin{folhadeaprovacao}
	\parindent=0pt
	\setlength{\ABNTEXsignskip}{1.5cm}

	Monografia de Graduação sob o título \textit{\imprimirtitulo} apresentada por {\imprimirautor} e aceita pela Faculdade de Tecnologia da {\imprimirinstituicao}, sendo aprovada por todos os membros da banca examinadora abaixo especificada:

	\assinatura{\fontsize{12}{15}\selectfont {\imprimirorientador} \\ \fontsize{11}{15} \selectfont {\fontsize{10}{12}\selectfont Instituto de Computação \par \imprimirinstituicao }}
	\vspace{1cm}
	\assinatura{Dr. Walter Esteves de Castro Júnior \\ {\fontsize{10}{12}\selectfont Departamento de Física \par \imprimirinstituicao}}
	\vspace{1cm}
	\assinatura{Nome do Membro 2 \\ {\fontsize{10}{12}\selectfont Departamento de Membro 2 \par \imprimirinstituicao}}
	\vfill
      
	\begin{center}
		\fontsize{12}{15}\selectfont
		\vspace*{0.5cm}
		\imprimirlocal, data de aprovação (por extenso).
		\vspace*{1cm}
	\end{center}
  
\end{folhadeaprovacao}

% Dedicatória
\begin{dedicatoria}
   \vspace*{\fill}
   \noindent
   \leftskip=7cm
   \textit{À minha mãe, mulher guerreira e inspiradora quem me deu forças para concluir esta graduação.}
   \vspace{5cm}
\end{dedicatoria}

% Agradecimentos
\begin{agradecimentos}

Agradecimentos dirigidos àqueles que contribuíram de maneira relevante à elaboração do trabalho, sejam eles pessoas ou mesmo organizações.

\end{agradecimentos}

% Epígrafe
\begin{epigrafe}
    \vspace*{\fill}
	\begin{flushright}
		\textit{Lorem ipsum}

		Autor
	\end{flushright}\vspace{4cm}
\end{epigrafe}

% ---
% RESUMOS
% ---

% resumo em português
\setlength{\absparsep}{18pt} % ajusta o espaçamento dos parágrafos do resumo
\begin{resumo}

	Este documento versa sobre o desenvolvimento de um aplicativo computacional para processar solicitações de isenção de taxa de inscrição em concursos públicos de acordo com as normativas do Ministério do Desenvolvimento Social do Brasil e os interesses da Comissão Permanente de Concursos da UFAM. O sistema, que opera coletivamente com o Sistema de Isenção de Taxa de Concurso, do Ministério do Desenvolvimento Social do Brasil, permite que um órgão gestor prepare dados pessoais de candidatos solicitantes de isenção de taxa de inscrição para envio ao sistema do Ministério do Desenvolvimento, e após o processamento de tais solicitações pelo sistema, gere editais de publicação e relatórios com o objetivo de garantir a lisura e transparência deste processo tão democrático. No desenvolvimento foi utilizada a linguagem de programação \textit{Java} e tecnologias de grande consolidação no mercado como o \textit{Jasper Reports}, para geração de relatórios e o \textit{Apache POI}, adicionando suporte a arquivos do \textit{Microsoft Excel}.

	\textit{Palavras-chave}: isenção de taxa de inscrição, concurso, Ministério do Desenvolvimento Social, Java.

\end{resumo}

% resumo em inglês
\begin{resumo}[Abstract]
 \begin{otherlanguage*}{english}
   This is the english abstract.

   \vspace{\onelineskip}
 
   \noindent 
   \textit{Keywords}: Keyword 1, Keyword 2, Keyword 3.
 \end{otherlanguage*}
\end{resumo}

% ---
% inserir lista de figuras
% ---
\pdfbookmark[0]{\listfigurename}{lof}
\listoffigures*
\cleardoublepage
% ---

% ---
% inserir lista de tabelas
% ---
\pdfbookmark[0]{\listtablename}{lot}
\listoftables*
\cleardoublepage
% ---

% ---
% inserir lista de abreviaturas e siglas
% ---
\begin{siglas}
  \item[COMPEC] Comissão Permanente de Concursos
  \item[MDS] Ministério do Desenvolvimento Social
  \item[PSConcursos] Sistema de Concursos da UFAM
  \item[SISTAC] Sistema de Isenção de Taxa de Concurso
  \item[UFAM] Universidade Federal do Amazonas
\end{siglas}
% ---

% ---
% inserir lista de símbolos
% ---
\begin{simbolos}
  \item[$ \lambda $] Lambda
\end{simbolos}
% ---

% ---
% inserir o sumario
% ---
\pdfbookmark[0]{\contentsname}{toc}
\tableofcontents*
\cleardoublepage
% ---

\textual

\chapter{Introdução}

	Com missão de cultivar o saber em todas as áreas do conhecimento por meio do ensino, pesquisa e da extensão, a Universidade Federal do Amazonas (UFAM) é uma das principais
	portas de entrada para o desenvolvimento pessoal e intelectual, contando com cerca de xxxx alunos e 3400 servidores distribuídos em seis \textit{campi} ao redor do Estado do Amazonas, em 2023.
	
	Tomando como objeto de estudo e inspiração para este trabalho, um setor específico desta universidade foi adotado: a Comissão Permanente de Concursos (COMPEC), que é um órgão
	suplementar responsável pela execução dos principais processos seletivos de graduação e concursos para provimento de cargos da universidade.
	
	Uma das tarefas mais democráticas e delicadas executadas por este setor é o processo de isenção de pagamento de taxa de inscrição em processos seletivos ou concursos.
	
	Atualmente a COMPEC, como qualquer outra entidade do poder executivo do Brasil, adota três tipos de categorias de isenção: por cadastro no Registro Brasileiro de Doadores Voluntários de Medula Óssea (REDOME), por comprovação de baixa renda e curso de nível médio de forma gratuita e por meio do Cadastro Único para Programas Sociais do Governo Federal (CadÚnico).
	
	Um dos desafios enfrentados pela COMPEC é a gerência e correto processamento das solicitações de isenção, de forma a não prejudicar os candidatos, tampouco a imagem da UFAM e do funcionalismo público. Para ilustrar, apenas em 2023 a COMPEC realizou 10 concursos, mobilizando ao total 39.289 candidatos, onde 8.353 deles tiveram isenção concedida.
	
	Com o intuito de automatizar e otimizar tal processo, este trabalho apresenta uma aplicação de computador capaz de analisar, processar e gerar relatórios e editais de publicação, tomando como objeto de estudo a categoria de isenção mais volumosa em termos de solicitação: a categoria via CadÚnico, regulamentada pelo decreto n{\textdegree} 6.593, de 2 de outubro de 2008.
	
	A aplicação, denominada \textit{IsenSys}, procura ainda fornecer uma interface simples e objetiva, com dicas e tratamentos de forma a instruir intuitivamente sua utilização ao usuário, tomando ainda como alicerce no seu desenvolvimento, os cinco princípios fundamentais da Administração Pública do Brasil: legalidade, impessoalidade, moralidade, publicidade e eficiência.

	\section{Objetivos}
	
		\subsection{Objetivo Geral}
		
		Este projeto possui como objetivo apresentar um aplicativo processador de solicitações de isenção de taxa de inscrição de acordo com a regulamentação do CadÚnico, de forma a permitir agilidade e acurácia nos resultados, por parte de uma unidade gestora do governo federal do Brasil.
		
		\subsection{Objetivos Específicos}
		
		\begin{itemize}
			
			\item Objetivo 1;
			\item Objetivo 2;
			\item Objetivo 3;
			\item Objetivo 4;
			
		\end{itemize}
		
	\section{Organização da Monografia}
	
		Nesta seção deve ser apresentado como está organizado o trabalho, sendo descrito, portanto, do que trata cada capítulo.


% ---
% Capítulo 2
% ---
\chapter{Capítulo 2}

	Este é o primeiro capítulo da parte central do trabalho, isto é, o
desenvolvimento, a parte mais extensa de todo o trabalho. Geralmente o
desenvolvimento é dividido em capítulos, cada um com seções e subseções,
cujo tamanho e número de divisões variam em função da natureza do
conteúdo do trabalho.

	Em geral, a parte de desenvolvimento é subdividida em três capítulos:

	\begin{itemize}
		\item \textit{referencial ou embasamento teórico} – texto no qual se deve apresentar os aspectos teóricos, isto é, os conceitos utilizados e a definição dos mesmos; nesta parte faz-se a revisão de literatura sobre o assunto, resumindo-se os resultados de estudos feitos por outros autores, cujas obras citadas e consultadas devem constar nas referências;
	
		\item \textit{metodologia do trabalho ou procedimentos metodológicos} – deve constar o instrumental, os métodos e as técnicas aplicados para a elaboração do trabalho;
	
		\item \textit{resultados} – devem ser apresentados, de forma objetiva, precisa e clara, tanto os resultados positivos quanto os negativos que foram obtidos com o desenvolvimento do trabalho, sendo feita uma discussão que consiste na avaliação circunstanciada, na qual se estabelecem relações, deduções e generalizações.
	\end{itemize}

	É recomendável que o número total de páginas referente à parte de desenvolvimento não ultrapasse 60 (sessenta) páginas.

	\section{Seção 1}

		Teste de figura:

		\begin{figure}[h!]
			\begin{center}
			    \includegraphics[scale=0.5]{abntex2/ufam-logo}
			\end{center}
			\caption{\label{fig_grafico}Logo da UFAM. Retirado da Internet}
		\end{figure}
		
		Continuação do texto.
		
	\section{Seção 2}
	
		Referenciamento da figura inserida na seção anterior: 2.1
		
	\section{Seção 3}
	
		Seção 3
		
	\section{Seção 4}
	
		Seção 4

% ---
% Capítulo 3
% ---
\chapter{Capítulo 3}

	Algumas regras devem ser observadas na redação da monografia:

	\begin{itemize}
		\item ser claro, preciso, direto, objetivo e conciso, utilizando frases curtas e evitando ordens inversas desnecessárias;
		
		\item construir períodos com no máximo duas ou três linhas, bem como parágrafos com cinco linhas cheias, em média, e no máximo oito (ou seja, não construir parágrafos e períodos muito longos, pois isso cansa o(s) leitor(es) e pode fazer com que ele(s) percam a linha de raciocínio desenvolvida);
		
		\item a simplicidade deve ser condição essencial do texto; a simplicidade do texto não implica necessariamente repetição de formas e frases desgastadas, uso exagerado de voz passiva (como será iniciado, será realizado), pobreza vocabular etc. Com palavras conhecidas de todos, é possível escrever de maneira original e criativa e produzir frases elegantes, variadas, fluentes e bem alinhavadas;
		
		\item adotar como norma a ordem direta, por ser aquela que conduz mais facilmente o leitor à essência do texto, dispensando detalhes irrelevantes e indo diretamente ao que interessa, sem rodeios (verborragias);

		\item não começar períodos ou parágrafos seguidos com a mesma palavra, nem usar repetidamente a mesma estrutura de frase;

		\item desprezar as longas descrições e relatar o fato no menor número possível de palavras;
		
		\item recorrer aos termos técnicos somente quando absolutamente indispensáveis e nesse caso colocar o seu significado entre parênteses (ou seja, não se deve admitir que todos os que lerão o trabalho já dispõem de algum conhecimento desenvolvido no mesmo);
		
		\item dispensar palavras e formas empoladas ou rebuscadas, que tentem
transmitir ao leitor mera ideia de erudição;

		\item não perder de vista o universo vocabular do leitor, adotando a seguinte
regra prática: nunca escrever o que não se diria;

		\item usar termos coloquiais ou de gíria com extrema parcimônia (ou mesmo
nem serem utilizados) e apenas em casos muito especiais, para não darem ao leitor a ideia de vulgaridade e descaracterizar o trabalho;

		\item ser rigoroso na escolha das palavras do texto, desconfiando dos
sinônimos perfeitos ou de termos que sirvam para todas as ocasiões;

		\item em geral, há uma palavra para definir uma situação;

		\item encadear o assunto de maneira suave e harmoniosa, evitando a
criação de um texto onde os parágrafos se sucedem uns aos outros
como compartimentos estanques, sem nenhuma fluência entre si;

		\item ter um extremo cuidado durante a redação do texto, principalmente
com relação às regras gramaticais e ortográficas da língua;

		\item geralmente todo o texto é escrito na forma impessoal do verbo, não se utilizando,
portanto, de termos em primeira pessoa, seja do plural ou do singular.

	\end{itemize}

	Continuação.
	
	\section{Seção 1}
	
		Teste de uma tabela:

		\begin{table}[htbp]
			\caption{Tabela sem sentido.}
			\label{tabela-ssentido}
			\begin{center}
			\begin{tabular}{|c|c|}
				\hline
				Título Coluna & Título Coluna \\
				1 & 2 \\
				\hline
				X & Y \\
				\hline
				X & W \\
				\hline
			\end{tabular}
			\end{center}
		\end{table}
		
	\section{Seção 2}
	
		Seção 2.

	\section{Seção 3}
	
		Seção 3.
		
% - - -
% Capítulo 4
% - - -
\chapter{Capítulo 4}

	\section{Seção 1}
	
		Teste de símbolo:
		
		$\lambda$
		
	\section{Seção 2}
	
		Teste de abreviaturas:
		
% - - -
% Capítulo 5
% - - -
\chapter{Capítulo 5}

	\section{Seção 1}
	
		Seção 1.
	
	\section{Seção 2}
	
		Alguns exemplos de citação:

		Na tese de Doutorado de Paquete, discute-se sobre algoritmos de busca local estocásticos aplicados a problemas de Otimização Combinatória considerando múltiplos objetivos. Por sua vez, o trabalho de, publicado nos anais do IEEE CEC de 2003, mostra uma técnica de arquivamento também empregada no desenvolvimento de algoritmos evolucionários multiobjetivo, trabalho esse posteriormente estendido para um capítulo de livro dos mesmos autores. Por m, no relatório técnico de Jaszkiewicz (1998), fala-se sobre um algoritmo genético híbrido para problemas multi- critério, enquanto no artigo de jornal de Lopez et al. (LÓPEZ-IBÁÑEZ; PAQUETE; STÜTZLE, 2006) trata-se do trade-o entre algoritmos genéticos e metodologias de busca local, também aplicados no contexto multicritério e relacionado de alguma forma ao trabalho de Jaszkiewicz (1998).

		Outros exemplos relacionados encontram-se em (SILBERSCHATZ; KORTH; SUDARSHAN, 2002) (livro), (TURAU, 2001) (referência da Web) e (AGRA, 2004) (dissertação de Mestrado).

		\subsection{Subseção 2.1}
	
			Seção 2.1
			
	\section{Seção 3}
	
		Seção 3
		
\chapter{Considerações finais}

	As considerações finais formam a parte final (fechamento) do texto, sendo dito de forma resumida (1) o que foi desenvolvido no presente trabalho e quais os resultados do mesmo, (2) o que se pôde concluir após o desenvolvimento bem como as principais contribuições do trabalho, e (3) perspectivas para o desenvolvimento de trabalhos futuros. O texto referente às considerações finais do autor deve salientar a extensão e os resultados da contribuição do trabalho e os argumentos utilizados estar baseados em dados comprovados e fundamentados nos resultados e na discussão do texto, contendo deduções lógicas correspondentes aos objetivos do trabalho, propostos inicialmente.

% ----------------------------------------------------------
% ELEMENTOS PÓS-TEXTUAIS
% ----------------------------------------------------------
\postextual
\Spacing{1.5}
% -----------------------------------------------------------------------------
% Referencias Bibliograficas
% -----------------------------------------------------------------------------
\bibliography{experimental}

% ----------------------------------------------------------
% Apêndices
% ----------------------------------------------------------

% ---
% Inicia os apêndices
% ---
\begin{apendicesenv}

% ----------------------------------------------------------
\chapter{Primeiro apêndice}
% ----------------------------------------------------------

Os apêndices são textos ou documentos elaborados pelo autor, a fim de complementar sua argumentação, sem prejuízo da unidade nuclear do trabalho.

\end{apendicesenv}

% Anexos
% ----------------------------------------------------------

% ---
% Inicia os anexos
% ---
\begin{anexosenv}

% ---
\chapter{Primeiro anexo.}
% ---

Os anexos são textos ou documentos não elaborados pelo autor, que servem de fundamentação, comprovação e ilustração.

\end{anexosenv}

\end{document}