\documentclass[
% -- opções da classe memoir --
12pt,				% tamanho da fonte
openright,			% capítulos começam em pág ímpar (insere página vazia caso preciso)
twoside,			% para impressão em recto e verso. Oposto a oneside
a4paper,			% tamanho do papel. 
% -- opções da classe abntex2 --
%chapter=TITLE,		% títulos de capítulos convertidos em letras maiúsculas
%section=TITLE,		% títulos de seções convertidos em letras maiúsculas
%subsection=TITLE,	% títulos de subseções convertidos em letras maiúsculas
%subsubsection=TITLE,% títulos de subsubseções convertidos em letras maiúsculas
% -- opções do pacote babel --
english,			% idioma adicional para hifenização
brazil				% o último idioma é o principal do documento
]{abntex2}


% ---
% PACOTES
% ---

% ---
% Pacotes fundamentais 
% ---
\usepackage{lmodern}			% Usa a fonte Latin Modern
\usepackage[T1]{fontenc}		% Selecao de codigos de fonte.
\usepackage[utf8]{inputenc}		% Codificacao do documento (conversão automática dos acentos)
\usepackage{indentfirst}		% Indenta o primeiro parágrafo de cada seção.
\usepackage{color}				% Controle das cores
\usepackage{graphicx}			% Inclusão de gráficos
\usepackage{microtype} 			% para melhorias de justificação
% ---

% ---
% Pacotes adicionais, usados no anexo do modelo de folha de identificação
% ---
\usepackage{multicol}
\usepackage{multirow}
% ---

% ---
% Pacotes de citações
% ---
\usepackage[brazilian,hyperpageref]{backref}	 % Paginas com as citações na bibl
\usepackage[alf]{abntex2cite}	% Citações padrão ABNT

% --- 
% CONFIGURAÇÕES DE PACOTES
% --- 

% ---
% Configurações do pacote backref
% Usado sem a opção hyperpageref de backref
\renewcommand{\backrefpagesname}{Citado na(s) página(s):~}
% Texto padrão antes do número das páginas
\renewcommand{\backref}{}
% Define os textos da citação
\renewcommand*{\backrefalt}[4]{
	\ifcase #1 %
	Nenhuma citação no texto.%
	\or
	Citado na página #2.%
	\else
	Citado #1 vezes nas páginas #2.%
	\fi}%
% ---

% ---
% Informações de dados para CAPA e FOLHA DE ROSTO
% ---
\titulo{Aqui vai o título do projeto}
\autor{Felipe André Souza da Silva}
\local{Manaus - AM}
\data{\today}
\instituicao{%
	Universidade Federal do Amazonas - UFAM
	\par
	Faculdade de Tecnologia - FT
	\par
	Engenharia da Computação - FT05}
\tipotrabalho{Relatório técnico}
% O preambulo deve conter o tipo do trabalho, o objetivo, 
% o nome da instituição e a área de concentração 
\preambulo{Modelo canônico de Relatório Técnico e/ou Científico em conformidade
	com as normas ABNT apresentado à comunidade de usuários \LaTeX.}
	
% ---
% Configurações de aparência do PDF final

% alterando o aspecto da cor azul
\definecolor{blue}{RGB}{41,5,195}

% informações do PDF
\makeatletter
\hypersetup{
	%pagebackref=true,
	pdftitle={\@title}, 
	pdfauthor={\@author},
	pdfsubject={\imprimirpreambulo},
	pdfcreator={LaTeX with abnTeX2},
	pdfkeywords={abnt}{latex}{abntex}{abntex2}{relatório técnico}, 
	colorlinks=true,       		% false: boxed links; true: colored links
	linkcolor=blue,          	% color of internal links
	citecolor=blue,        		% color of links to bibliography
	filecolor=magenta,      		% color of file links
	urlcolor=blue,
	bookmarksdepth=4
}
\makeatother
% --- 

% --- 
% Espaçamentos entre linhas e parágrafos 
% --- 

% O tamanho do parágrafo é dado por:
\setlength{\parindent}{1.3cm}

% Controle do espaçamento entre um parágrafo e outro:
\setlength{\parskip}{0.2cm}  % tente também \onelineskip

% ---
% compila o indice
% ---
\makeindex
% ---

% ----
% Início do documento
% ----
\begin{document}
	
% Seleciona o idioma do documento (conforme pacotes do babel)
%\selectlanguage{english}
\selectlanguage{brazil}

% Retira espaço extra obsoleto entre as frases.
\frenchspacing 
	
	\imprimircapa
	
	% ---
	% Folha de rosto
	% (o * indica que haverá a ficha bibliográfica)
	% ---
	\imprimirfolhaderosto*
	% ---
	
	% ---
	% Anverso da folha de rosto:
	% ---
	
	{
		\ABNTEXchapterfont
		
		\vspace*{\fill}
		
		Conforme a ABNT NBR 10719:2015, seção 4.2.1.1.1, o anverso da folha de rosto
		deve conter:
		
		\begin{alineas}
			\item nome do órgão ou entidade responsável que solicitou ou gerou o
			relatório; 
			\item título do projeto, programa ou plano que o relatório está relacionado;
			\item título do relatório;
			\item subtítulo, se houver, deve ser precedido de dois pontos, evidenciando a
			sua subordinação ao título. O relatório em vários volumes deve ter um título
			geral. Além deste, cada volume pode ter um título específico; 
			\item número do volume, se houver mais de um, deve constar em cada folha de
			rosto a especificação do respectivo volume, em algarismo arábico; 
			\item código de identificação, se houver, recomenda-se que seja formado
			pela sigla da instituição, indicação da categoria do relatório, data,
			indicação do assunto e número sequencial do relatório na série; 
			\item classificação de segurança. Todos os órgãos, privados ou públicos, que
			desenvolvam pesquisa de interesse nacional de conteúdo sigiloso, devem
			informar a classificação adequada, conforme a legislação em vigor; 
			\item nome do autor ou autor-entidade. O título e a qualificação ou a função
			do autor podem ser incluídos, pois servem para indicar sua autoridade no
			assunto. Caso a instituição que solicitou o relatório seja a mesma que o
			gerou, suprime-se o nome da instituição no campo de autoria; 
			\item local (cidade) da instituição responsável e/ou solicitante; NOTA: No
			caso de cidades homônimas, recomenda-se o acréscimo da sigla da unidade da
			federação.
			\item ano de publicação, de acordo com o calendário universal (gregoriano),
			deve ser apresentado em algarismos arábicos.
		\end{alineas}
		
		\vspace*{\fill}
	}

% ---
% Agradecimentos
% ---
\begin{agradecimentos}
	O agradecimento principal é direcionado a Youssef Cherem, autor do
	\nameref{formulado-identificacao} (\autopageref{formulado-identificacao}).
	
	Os agradecimentos especiais são direcionados ao Centro de Pesquisa em
	Arquitetura da Informação\footnote{\url{http://www.cpai.unb.br/}} da Universidade de
	Brasília (CPAI), ao grupo de usuários
	\emph{latex-br}\footnote{\url{http://groups.google.com/group/latex-br}} e aos
	novos voluntários do grupo
	\emph{\abnTeX}\footnote{\url{http://groups.google.com/group/abntex2} e
		\url{http://www.abntex.net.br/}}~que contribuíram e que ainda
	contribuirão para a evolução do abn\TeX.
	
\end{agradecimentos}
% ---

% ---
% RESUMO
% ---

% resumo na língua vernácula (obrigatório)
\setlength{\absparsep}{18pt} % ajusta o espaçamento dos parágrafos do resumo
\begin{resumo}
	Segundo a \citeonline[3.1-3.2]{NBR6028:2003}, o resumo deve ressaltar o
	objetivo, o método, os resultados e as conclusões do documento. A ordem e a extensão
	destes itens dependem do tipo de resumo (informativo ou indicativo) e do
	tratamento que cada item recebe no documento original. O resumo deve ser
	precedido da referência do documento, com exceção do resumo inserido no
	próprio documento. (\ldots) As palavras-chave devem figurar logo abaixo do
	resumo, antecedidas da expressão Palavras-chave:, separadas entre si por
	ponto e finalizadas também por ponto.
	
	\noindent
	\textbf{Palavras-chaves}: latex. abntex. editoração de texto.
\end{resumo}
% ---
	
% ---
% inserir lista de ilustrações
% ---
\pdfbookmark[0]{\listfigurename}{lof}
\listoffigures*
\cleardoublepage	
	
	% ---
	% inserir lista de tabelas
	% ---
	\pdfbookmark[0]{\listtablename}{lot}
	\listoftables*
	\cleardoublepage
	% ---

% ---
% inserir lista de abreviaturas e siglas
% ---
\begin{siglas}
	\item[ABNT] Associação Brasileira de Normas Técnicas
	\item[abnTeX] ABsurdas Normas para TeX
\end{siglas}
% ---

% ---
% inserir lista de símbolos
% ---
\begin{simbolos}
	\item[$ \Gamma $] Letra grega Gama
	\item[$ \Lambda $] Lambda
	\item[$ \zeta $] Letra grega minúscula zeta
	\item[$ \in $] Pertence
\end{simbolos}
% ---

% ---
% inserir o sumario
% ---
\pdfbookmark[0]{\contentsname}{toc}
\tableofcontents*
\cleardoublepage
% ---

% ----------------------------------------------------------
% ELEMENTOS TEXTUAIS
% ----------------------------------------------------------
\textual

% ----------------------------------------------------------
% Introdução (exemplo de capítulo sem numeração, mas presente no Sumário)
% ----------------------------------------------------------
\chapter*[Introdução]{Introdução}
\addcontentsline{toc}{chapter}{Introdução}

Este documento e seu código-fonte são exemplos de referência de uso da classe
\textsf{abntex2} e do pacote \textsf{abntex2cite}. O documento 
exemplifica a elaboração de relatórios técnicos e/ou científicos produzidos
conforme a ABNT NBR 10719:2015 \emph{Informação e documentação - Relatório
	técnico e/ou científico - Apresentação}.

A expressão ``Modelo canônico'' é utilizada para indicar que \abnTeX\ não é
modelo específico de nenhuma universidade ou instituição, mas que implementa tão
somente os requisitos das normas da ABNT. Uma lista completa das normas
observadas pelo \abnTeX\ é apresentada em \citeonline{abntex2classe}.

Sinta-se convidado a participar do projeto \abnTeX! Acesse o site do projeto em
\url{http://www.abntex.net.br/}. Também fique livre para conhecer,
estudar, alterar e redistribuir o trabalho do \abnTeX, desde que os arquivos
modificados tenham seus nomes alterados e que os créditos sejam dados aos
autores originais, nos termos da ``The \LaTeX\ Project Public
License''\footnote{\url{http://www.latex-project.org/lppl.txt}}.

Encorajamos que sejam realizadas customizações específicas deste exemplo para
universidades e outras instituições --- como capas, folhas de rosto, etc.
Porém, recomendamos que ao invés de se alterar diretamente os arquivos do
\abnTeX, distribua-se arquivos com as respectivas customizações.
Isso permite que futuras versões do \abnTeX~não se tornem automaticamente
incompatíveis com as customizações promovidas. Consulte
\citeonline{abntex2-wiki-como-customizar} para mais informações.

Este documento deve ser utilizado como complemento dos manuais do \abnTeX\ 
\cite{abntex2classe,abntex2cite,abntex2cite-alf} e da classe \textsf{memoir}
\cite{memoir}. 

Equipe \abnTeX 

Lauro César Araujo



\end{document}